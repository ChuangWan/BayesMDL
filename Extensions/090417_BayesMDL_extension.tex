\documentclass[12pt]{article}
%\documentclass[jasa,12pt]{article}
%\documentclass[leqno,12pt]{article}
\usepackage{mathrsfs}
\usepackage{amsfonts,amsmath,latexsym,amsthm}
\usepackage{graphicx} 
\usepackage{xr-hyper}
\usepackage{url, hyperref}
\usepackage{setspace}
\usepackage{natbib}
\bibpunct{(}{)}{;}{a}{}{,}
%\usepackage{fmtcount}
%\usepackage{longtable}  %% allow tables to flow over page boundaries
\usepackage{multirow}  %% allow multi row in tables (see first table in Section 6.1)
%\usepackage{enumerate}  %% allow different style of counter in enumerate environment (see asymptotic theorems)
%\usepackage{rotating}  %% allow rotation of letters (see tables in simulation)
%\usepackage[latin1]{inputenc} % From LaTeX distribution
%\usepackage{calc}         % From LaTeX distribution 
%\usepackage{ifthen}       % From LaTeX distribution
%\usepackage{pst-all}      % From PSTricks
%\usepackage{pst-poly}     % From pstricks/contrib/pst-poly
%\usepackage{multido}      % From PSTricks
%\input{random.tex}        % From CTAN/macros/generic
%\usepackage[notref, notcite]{showkeys} % to show ref labels

\usepackage{enumitem}


\hypersetup{
    colorlinks,%
    citecolor=blue,%
    filecolor=black,%
    linkcolor=blue,%
    urlcolor=blue
}


\pdfminorversion=4


\doublespacing
% DON'T change margins - should be 1 inch all around.
%\addtolength{\oddsidemargin}{-.5in}%
%\addtolength{\evensidemargin}{-.5in}%
%\addtolength{\textwidth}{1in}%
%\addtolength{\textheight}{1.3in}%
%\addtolength{\topmargin}{-.8in}%

\setlength{\textwidth}{16.7cm} \setlength{\textheight}{22.5cm}
\setlength{\oddsidemargin}{-0.1cm} \setlength{\evensidemargin}{-1cm}
\setlength{\topmargin}{-1.5cm}
\usepackage{setspace}


\newtheorem{theorem}{Theorem}%[section]
\newtheorem{corollary}{Corollary}%[section]
\newtheorem{lemma}{Lemma}%[section]
\newtheorem{proposition}{Proposition}%[section]
\newtheorem{definition}{Definition}%[section]
\newtheorem{remark}{Remark}%[section]
\newtheorem{problem}{Problem}
\newtheorem{counterexample}{Counterexample}[section]
%\newtheorem*{assumption}{Assumption}
\newtheorem{assumption}{Assumption}
\newtheorem{example}{Example}
%\numberwithin{equation}{section}

\def\TODO{\textcolor{red}}
\def\red{\textcolor{red}}
\def\blue{\textcolor{blue}}


%%%%%%%%%%%%%%%%%  blinding %%%%%%%%%%%%%%%%%%%%%%%%%%%%%%
%\long\gdef\blind#1#2{\ifbld{\em \color{blue}#1}\else{#2}\fi} %% Text for blinding shown in blue
\long\gdef\blind#1#2{\ifbld{#1}\else{#2}\fi} %% Text for blinding shown in black
\newif\ifbld \bldfalse
%\newif\ifbld \bldtrue
%%%%%%%%%%%%%%%%%  blinding %%%%%%%%%%%%%%%%%%%%%%%%%%%%%%

%\newpsobject{showgrid}{psgrid}{subgriddiv=1,griddots=10,gridlabels=6pt}



%macro file 
%\input{macros}



\newpage



\title{Extension of Bayesian MDL for Changepoint Detection} \date{}
\author{\blind{}{Yingbo Li\footnote{ e-mail: {\tt carolli13@gmail.com}.}}
       }

\begin{document}

\maketitle

\externaldocument{appendix}


%%%%%%%%%%%%%%%%%%%%%%%%%%%%%%%%%%%%%%%%%%%%%%%%%%%%
% abstract
%%%%%%%%%%%%%%%%%%%%%%%%%%%%%%%%%%%%%%%%%%%%%%%%%%%%
%\begin{abstract}
%
%
%
%\end{abstract}
%
%\noindent%
%{\it Keywords:}  Bayesian model selection, 



%\newpage\tableofcontents %% to show outline, too be removed later

%%%%%%%%%%%%%%%%%%%%%%%%%%%%%%%%%%%%%%%%%%%%%%%%%%%%%
% Draft
%%%%%%%%%%%%%%%%%%%%%%%%%%%%%%%%%%%%%%%%%%%%%%%%%%%%%
\section{Current Bayesian MDL Formula}

\subsection{Linear model}

Given a changepoint model $\boldsymbol{\eta}$, the sampling distribution 
\eqref{eq:likelihood1} has the regression representation
\begin{equation}
\label{eq:likelihood3}
\mathbf{X}_{1:N} = \mathbf{A}_{1:N} \mathbf{s} + 
\mathbf{D}_{1:N}\boldsymbol\mu + \boldsymbol\epsilon_{1:N},
\end{equation}
with $\mathbf{A}_{1:N}\in \mathbb{R}^{N \times T}$ and 
$\mathbf{D}_{1:N} \in \mathbb{R}^{N \times m}$ as seasonal and regime 
indicator matrices, respectively:
\begin{align*}
&\left[ \mathbf{A}_{1:N} \right]_{t,v} = 
	\mathbf{1}( \text{time } t \text{ is in season } v ), ~~~ v = 1,  \ldots, T,\\
&\left[ \mathbf{D}_{1:N} \right]_{t,r-1} = 
	\mathbf{1}( \text{time } t \text{ is in regime } r ), ~~~ r = 2, \ldots, m + 1,
\end{align*}
where $\mathbf{1}(A)$ denotes the indicator of the event $A$. 



%%%%%%%%%%%%%%%%%%%%%%%%%%%%%%%%%%%%%%%%%%%%%%%%%%%%%
%\section*{Supplementary Materials}
%\begin{description}[leftmargin=*, itemsep=0em]
%\item[Appendix \ref{appendix:proofs}:] a list of assumptions, all the proofs, and some additional theoretical results.
%\item[CHIC\_examples.zip:] R scripts for producing the simulation and real data results.
%
%\end{description}

%{\bf R package}: R package  ``gglm'' for implementation of all methods used in the paper, including the CH-$g$, the Beta-prime,
%and the Robust priors. (gglm\_1.0.tar.gz) \\
 
 %%%%%%%%%%%%%%%%%%%%%%%%%%%%%%%%%%%%%%%%%%%%%%%%%%%%%%%%
%\blind{}{
%\section*{Acknowledgement}
%The authors thank 
%}

%%%%%%%%%%%%%%%%%%%%%%%%%%%%%%%%%%%%%%%%%%%%%%%%%%%%%%%%
{\small
\singlespacing
\bibliographystyle{asa}
%\bibliographystyle{agsm}
\bibliography{/Users/yingbo/Dropbox/Research/MyRef.bib}

}
%%%%%%%%%%%%%%%%%%%%%%%%%%%%%%%%%%%%%%%%%%%%%%%%%%%%%%%%
%\pagebreak
%
%\setcounter{section}{0}
%\setcounter{page}{1}
%
%
%\begin{center}
%{\LARGE
%Mixtures of $g$-priors in Generalized Linear Models\\
%}
%
%{\LARGE
%\emph{
%Supplementary Materials
%}
%}
%\end{center}
%\input{appendix.tex}

\end{document}
